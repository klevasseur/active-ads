%% AMS-LaTeX Created with the Wolfram Language : www.wolfram.com

\documentclass{article}
\usepackage{amsmath, amssymb, graphics, setspace}

\newcommand{\mathsym}[1]{{}}
\newcommand{\unicode}[1]{{}}

\newcounter{mathematicapage}
\begin{document}

\title{MATH.3220 Practice Exam $\#$1}
\author{}
\date{}
\maketitle



Instructions: { }When you are asked to explain something, do so. Simple yes/no answers will not get full credit.



1. Consider the following directed graph. What is the adjacency matrix of this graph? 

\begin{doublespace}
\noindent\(\)
\end{doublespace}

\begin{doublespace}
\noindent\(\left(
\begin{array}{cccccc}
 0 & 0 & 0 & 0 & 0 & 1 \\
 0 & 0 & 0 & 0 & 0 & 0 \\
 1 & 0 & 0 & 0 & 1 & 0 \\
 0 & 0 & 1 & 0 & 0 & 0 \\
 0 & 0 & 1 & 1 & 0 & 1 \\
 0 & 1 & 0 & 1 & 1 & 0 \\
\end{array}
\right)\)
\end{doublespace}



2. { } (a) For any undirected graph \(G\), define what is meant by the diameter, radius, and center of G.



Let \(d(v)\) be the longest distance in \(G\) from \(v\) to the other vertices in \(G\), { }where distance is measured by the number of edges between
vertices. 



The diameter is the largest value \(d(v)\).



The radius is the smallest value of \(d(v)\).



The center is \(\{v | d(v)=\text{the} \text{radius} \text{of} G\}\).



 { } { }(b) { } Apply your definitions to determine the diameter, radius, and center of the graph with the following distance matrix, where \(D_{i,j}\)
the length of the shortest path from \(i\) to \(j\).
graph_measures.png
\begin{doublespace}
\noindent\(\text{/Users/ken$\_$l/Library/Mobile Documents/com$\sim $apple$\sim $CloudDocs/MATH.3220}\)
\end{doublespace}

\begin{doublespace}
\noindent\(\pmb{\text{SeedRandom}[69];}\\
\pmb{n = 12; e = 17;}\\
\pmb{(\text{DG}=\text{RandomGraph}[\{n, e\}, \text{EdgeLabels} \text{-$>$} \text{None}, \text{EdgeLabelStyle} \text{-$>$} \text{Larger}, }\\
\pmb{\text{VertexLabelStyle} \text{-$>$} \text{Directive}[\text{Italic}, 18], \text{VertexSize} \text{-$>$} \text{Large}, }\\
\pmb{\text{VertexStyle} \text{-$>$} \text{White}, }\\
\pmb{\text{VertexLabels} \text{-$>$} \text{Table}[i \text{-$>$} \text{Placed}[\text{ToString}[i], \text{Center}], \{i, 1, n\}]])}\\
\pmb{\text{DG}\text{//}\text{Export}[\text{{``}fig-compute-distance-graph.png{''}},\#,\text{{``}PNG{''}}]\&}\\
\pmb{\text{Dee}=\text{GraphDistanceMatrix}[\text{DG}]}\)
\end{doublespace}

\begin{doublespace}
\noindent\(\)
\end{doublespace}

\begin{doublespace}
\noindent\(\text{fig-compute-distance-graph.png}\)
\end{doublespace}

\begin{doublespace}
\noindent\(\left(
\begin{array}{cccccccccccc}
 0 & 2 & 2 & 2 & 3 & 1 & 1 & 3 & 3 & 1 & 2 & 2 \\
 2 & 0 & 3 & 3 & 2 & 2 & 1 & 4 & 1 & 2 & 2 & 1 \\
 2 & 3 & 0 & 2 & 5 & 3 & 2 & 3 & 4 & 1 & 4 & 3 \\
 2 & 3 & 2 & 0 & 3 & 1 & 2 & 1 & 3 & 1 & 2 & 3 \\
 3 & 2 & 5 & 3 & 0 & 2 & 3 & 4 & 1 & 4 & 1 & 3 \\
 1 & 2 & 3 & 1 & 2 & 0 & 1 & 2 & 2 & 2 & 1 & 2 \\
 1 & 1 & 2 & 2 & 3 & 1 & 0 & 3 & 2 & 1 & 2 & 1 \\
 3 & 4 & 3 & 1 & 4 & 2 & 3 & 0 & 4 & 2 & 3 & 4 \\
 3 & 1 & 4 & 3 & 1 & 2 & 2 & 4 & 0 & 3 & 1 & 2 \\
 1 & 2 & 1 & 1 & 4 & 2 & 1 & 2 & 3 & 0 & 3 & 2 \\
 2 & 2 & 4 & 2 & 1 & 1 & 2 & 3 & 1 & 3 & 0 & 3 \\
 2 & 1 & 3 & 3 & 3 & 2 & 1 & 4 & 2 & 2 & 3 & 0 \\
\end{array}
\right)\)
\end{doublespace}

\begin{doublespace}
\noindent\(\pmb{\text{Max}[\text{Dee}]}\)
\end{doublespace}



 The diameter is 4, the radius is 2, and the center is \(\{4,11\}\).



3. { } { }Consider the following undirected graph.

\begin{doublespace}
\noindent\(\pmb{\text{SeedRandom}[34];}\\
\pmb{n = 12; e = 17;}\\
\pmb{G=\text{RandomGraph}[\{n, e\}, \text{VertexLabelStyle} \text{-$>$} \text{Directive}[\text{Italic}, 18], }\\
\pmb{\text{VertexSize} \text{-$>$} \text{Large}, \text{VertexStyle} \text{-$>$} \text{White}, }\\
\pmb{\text{VertexLabels} \text{-$>$} \text{Table}[i \text{-$>$} \text{Placed}[\text{ToString}[i], \text{Center}], \{i, 1, n\}]]}\)
\end{doublespace}

\begin{doublespace}
\noindent\(\)
\end{doublespace}



$\quad $(a) { }Determine the degree sequence of the graph



 The degrees, in order of the vertex numbers are \(3,4,1,1,3,2,3,2,3,3,4, \text{and} 5\) and so the degree sequence is \(5,4,4,3,3,3,3,3,2,2,1,1\).



$\quad $(b) { }In a breadth-first search of this graph starting at vertex 8, what would be the depth sets?



\(D_0= \{8\}, D_1=\{1,11\}, D_2=\{7,12\}, D_3=\{6,2,10\}, D_4=\{4,5,9\}, D_5=\{3\}\)



 4. Consider the following graph. { }



$\quad $(a) Is this graph Hamiltonian? { }Explain your answer.



Yes because there is a Hamiltonian circuit. { }One such is \(1,4,8,2,7,5,6 ,1\).



$\quad $(b) { }This graph is not Eulerian. { }Explain why, and identify what edges could be added and/or removed to the graph to make it Eulerian.



The degrees of the vertices, in order of vertex number, are 4, 6, 3, 5, 5, 6, 3. { } There are four vertices of odd degree and so no Eulerian circuit.
{ } We can adjust the graph to be Eulerian by removing or adding edges between pairs of vertices of odd degree. { }For example, we could remove the
edges $\{$3,7$\}$ and { }\(\{4,5\}\), or we could add edges $\{$3,5$\}$ and $\{$4,7$\}$. { }Either way, we end up with a { }connected graph where
all vertices have even degree.

\begin{doublespace}
\noindent\(\pmb{\text{SeedRandom}[2987];n=8;}\\
\pmb{J=\text{RandomGraph}[\{n,18\}];}\\
\pmb{\text{While}[\text{Length}[\text{Select}[\text{VertexDegree}[J],\text{OddQ}]]>4,}\\
\pmb{J=\text{RandomGraph}[\{n,18\},\text{VertexLabelStyle}\to \text{Large},\text{VertexSize} \text{-$>$} \text{Large}, }\\
\pmb{\text{VertexStyle} \text{-$>$} \text{White}, }\\
\pmb{\text{VertexLabels} \text{-$>$} \text{Table}[i \text{-$>$} \text{Placed}[\text{ToString}[i], \text{Center}], \{i, 1, 8\}]]];}\\
\pmb{\text{GraphPlot}[J,\text{VertexLabelStyle}\to \text{Large},\text{VertexSize} \text{-$>$} \text{Large}, }\\
\pmb{\text{VertexStyle} \text{-$>$} \text{White}, }\\
\pmb{\text{VertexLabels} \text{-$>$} \text{Table}[i \text{-$>$} \text{Placed}[\text{ToString}[i], \text{Center}], \{i, 1, 8\}]]}\)
\end{doublespace}

\begin{doublespace}
\noindent\(\)
\end{doublespace}







5. { }State the theorem that relates the numbers of vertices and edges of a tree.



$\quad $See Theorem 10.1.11, part 5. { }



6. { }Define: Spanning Tree of a graph



Assuming \textit{ G} is connected, a spanning subtree of \textit{ G} is a graph with the same vertices as \textit{ G} and a subset of edges of \textit{
G} that forms a tree. { }See Definition 10.2.2.



7. { }Use Prim{'}s algorithm starting at vertex 1 to find a minimal spanning tree for the following graph.

\begin{doublespace}
\noindent\(\pmb{\text{SeedRandom}[17];}\\
\pmb{n = 10; e = 13;}\\
\pmb{g=\text{RandomGraph}[\{n, e\}, \text{EdgeWeight} \text{-$>$} \text{RandomInteger}[\{5, 15\}, e], }\\
\pmb{\text{EdgeLabels} \text{-$>$} \text{{``}EdgeWeight{''}}, \text{EdgeLabelStyle} \text{-$>$} \text{Larger}, }\\
\pmb{\text{VertexLabelStyle} \text{-$>$} \text{Directive}[\text{Italic}, 18], \text{VertexSize} \text{-$>$} \text{Medium}, }\\
\pmb{\text{VertexStyle} \text{-$>$} \text{White}, }\\
\pmb{\text{VertexLabels} \text{-$>$} \text{Table}[i \text{-$>$} \text{Placed}[\text{ToString}[i], \text{Center}], \{i, 1, n\}]]}\)
\end{doublespace}

\begin{doublespace}
\noindent\(\)
\end{doublespace}



The edges that are added, in order are \(\{1,5\}, \{1,9\}, \{1,2\},\{9,7\}, \{7,6\},\{7,3\},\{1,10\},\{5,8\},\{9,4\}\). { }This assumes that if there
is a tie in possibly adding an edge, the smallest vertex number is added, which is the case, when \(\{1,2\}\) is added.



8. Build an expression tree for the expression { }\(x+(y\cdot (z+w\cdot t))\) and then list the vertices in a preorder traversal of the tree. 



Doing this by hand, you would normally use addition and multiplication symbols instead of the words Plus and Times.

\begin{doublespace}
\noindent\(\pmb{\text{TreeForm}[x+(y(z+w t))]}\)
\end{doublespace}

\begin{doublespace}
\noindent\(\)
\end{doublespace}



9. Consider the { }operation \(\square\) on the set S =\(\text{the} \text{positive} \text{integers} ,( \text{NOT}\{1,2,3,4,5\})\), defined by { }\(a\square
b\text{  }= a+b-1\). { }\\
$\quad $(a) { }Is \(\square\) associative on \(S\)



\((a\square b)\square c= (a+b-1)\square c = a+b-1+c-1 = a+b+c-2\\
\\
a\square (b\square c)= a\square (b+c-1) = a+(b+c-1)-1 = a+b+c-2\)



Same result, so \(\square\) is associative.



$\quad $(b) { }Is \(\square\) { }commutative on \(S\)



Yes, { }by comparing \(a\square b\) and \(b\square a\), you see that the result is also the same here.



$\quad $(c) Does { }\(\square\) have an identity in \(S\) ?



If \(e\) is an identity, { }we would need \(a\square e=a+e-1 = a\). { }Solving \(a+e-1 = a\) for \(e\), we see that \(e=1\) is an identity. { }



$\quad $(d) { }Does { }\(\square\) have the inverse property on \(S\)?



No, the inverse of { }\(a\in S\), { }would be \(b\in S\) { }such that \(a\square b=1\). { } Therefore \(a+b-1=1\), or \(b=-a+2\). { }But since we
assume \(a\) is a positive integer \(b \notin S\) unless \(a = 1\), { }and so no other elements have an inverse.



10. { }Define the following terms in clear English sentences.



See sections 11.1 and 11.2 for these definitions.



$\quad $(a) { } { }Assume \(*\) is a binary operation on a set \(S\). { } A subset, \(T\), of \(S\) is closed with respect to \(*\)...



$\quad $(c) { }\([G,*]\) is an abelian group if ...



11. Let \(A\) be a nonempty set and \(\mathcal{P}(A)\) the set of all subsets of \(A\). Consider the algebraic system \([\mathcal{P}(A),\cup ]\)



(a) What is the identity for this system?



$\quad \quad $The empty set, since { } \(B \cup \emptyset =B=\emptyset \cup B\) { }for all \(B\subseteq A\).



(b) For what sets \(X\in \mathcal{P}(A)\) is there an inverse?



Only the empty set has an inverse. { } If \(B\) is nonempty, and C is any set, \(B \subseteq B\cup C\) and so \(B\cup C \neq \emptyset\).



12. The real numbers with either addition or multiplication is a monoid. With which operation is it also a group. { }Explain your answer. 



With addition, the real number form a groups since every real number has an additive inverse. { } With addition, you don{'}t get a group because
0 has no multiplicative inverse. { } That can be {``}fixed{''} by removing 0 to give us the group \(\left[\mathbb{R}^*; \cdot \right]\).


\section{}



4. { } Use only the axioms of a group to prove that \([G,*]\) is a group, and \(a,b\in  G\), then



$\quad \quad $\(x*a= b \Longrightarrow  x=b*a^{-1}\)



5. { }(a) { }Find integers \(s\) and \(t\) such that \(91x + 27 y = \gcd (91,27)\)























































(b) { }What are the additive and multiplicative inverses of 27 for modular addition and multiplication modulo 91?



















6. How many elements are there in the two groups { }\(\mathbb{Z}_{25}\) and { }\(U_{25}\)? { }Explain how you arrived at your answers.


\section{Code}

\begin{doublespace}
\noindent\(\pmb{\text{SeedRandom}[7];}\\
\pmb{G=\text{RandomGraph}[\{6,10\},\text{DirectedEdges}\to \text{True},\text{VertexSize}\to \text{Medium},}\\
\pmb{\text{VertexStyle}\to \text{White},}\\
\pmb{\text{VertexLabels}\to \text{Table}[i\to \text{Placed}[\text{ToString}[i],\text{Center}],\{i,1,7\}],}\\
\pmb{\text{VertexLabelStyle}\to \text{Larger}]}\)
\end{doublespace}

\begin{doublespace}
\noindent\(\)
\end{doublespace}

\begin{doublespace}
\noindent\(\pmb{\text{AdjacencyMatrix}[G]\text{//}\text{Normal}}\)
\end{doublespace}

\begin{doublespace}
\noindent\(\left(
\begin{array}{cccccc}
 0 & 0 & 0 & 0 & 0 & 1 \\
 0 & 0 & 0 & 0 & 0 & 0 \\
 1 & 0 & 0 & 0 & 1 & 0 \\
 0 & 0 & 1 & 0 & 0 & 0 \\
 0 & 0 & 1 & 1 & 0 & 1 \\
 0 & 1 & 0 & 1 & 1 & 0 \\
\end{array}
\right)\)
\end{doublespace}

\begin{doublespace}
\noindent\(\pmb{\text{SeedRandom}[67];}\\
\pmb{n = 12; e = 18;}\\
\pmb{\text{DG}=\text{RandomGraph}[\{n, e\}, \text{EdgeLabels} \text{-$>$} \text{{``}EdgeWeight{''}}, }\\
\pmb{\text{EdgeLabelStyle} \text{-$>$} \text{Larger}, \text{VertexLabelStyle} \text{-$>$} \text{Directive}[\text{Italic}, 18], }\\
\pmb{\text{VertexSize} \text{-$>$} \text{Large}, \text{VertexStyle} \text{-$>$} \text{White}, }\\
\pmb{\text{VertexLabels} \text{-$>$} \text{Table}[i \text{-$>$} \text{Placed}[\text{ToString}[i], \text{Center}], \{i, 1, n\}]]}\\
\pmb{\text{Dee}=\text{GraphDistanceMatrix}[\text{DG}]}\)
\end{doublespace}

\begin{doublespace}
\noindent\(\)
\end{doublespace}

\begin{doublespace}
\noindent\(\left(
\begin{array}{cccccccccccc}
 0 & 1 & 1 & 2 & 1 & 4 & 1 & 4 & 3 & 4 & 2 & 2 \\
 1 & 0 & 2 & 1 & 1 & 3 & 2 & 3 & 2 & 3 & 2 & 2 \\
 1 & 2 & 0 & 2 & 2 & 4 & 1 & 4 & 3 & 4 & 2 & 2 \\
 2 & 1 & 2 & 0 & 2 & 2 & 1 & 2 & 1 & 2 & 1 & 1 \\
 1 & 1 & 2 & 2 & 0 & 3 & 2 & 3 & 2 & 3 & 1 & 3 \\
 4 & 3 & 4 & 2 & 3 & 0 & 3 & 2 & 1 & 2 & 2 & 3 \\
 1 & 2 & 1 & 1 & 2 & 3 & 0 & 3 & 2 & 3 & 1 & 1 \\
 4 & 3 & 4 & 2 & 3 & 2 & 3 & 0 & 1 & 2 & 2 & 3 \\
 3 & 2 & 3 & 1 & 2 & 1 & 2 & 1 & 0 & 1 & 1 & 2 \\
 4 & 3 & 4 & 2 & 3 & 2 & 3 & 2 & 1 & 0 & 2 & 3 \\
 2 & 2 & 2 & 1 & 1 & 2 & 1 & 2 & 1 & 2 & 0 & 2 \\
 2 & 2 & 2 & 1 & 3 & 3 & 1 & 3 & 2 & 3 & 2 & 0 \\
\end{array}
\right)\)
\end{doublespace}

\begin{doublespace}
\noindent\(\pmb{\text{Manipulate}[\text{SeedRandom}[k];}\\
\pmb{n=9;e=13;}\\
\pmb{\text{RandomGraph}[\{n,e\},\text{VertexLabelStyle}\to \text{Directive}[\text{Italic},18],}\\
\pmb{\text{VertexSize}\to \text{Large},\text{VertexStyle}\to \text{White},}\\
\pmb{\text{VertexLabels}\to \text{Table}[i\to \text{Placed}[\text{ToString}[i],\text{Center}],\{i,1,n\}]],}\\
\pmb{\{k,20,100,1\}]}\)
\end{doublespace}

\begin{doublespace}
\noindent\(\)
\end{doublespace}

\begin{doublespace}
\noindent\(\pmb{\text{Manipulate}[\text{SeedRandom}[k];}\\
\pmb{n=7;e=10;}\\
\pmb{\text{RandomGraph}[\{n,e\},\text{EdgeWeight}\to \text{RandomInteger}[\{5,15\},e],}\\
\pmb{\text{EdgeLabels}\to \text{{``}EdgeWeight{''}},\text{EdgeLabelStyle}\to \text{Larger},}\\
\pmb{\text{VertexLabelStyle}\to \text{Directive}[\text{Italic},18],\text{VertexSize}\to \text{Large},}\\
\pmb{\text{VertexStyle}\to \text{White},}\\
\pmb{\text{VertexLabels}\to \text{Table}[i\to \text{Placed}[\text{ToString}[i],\text{Center}],\{i,1,n\}]],}\\
\pmb{\{k,20,100,1\}]}\)
\end{doublespace}

\begin{doublespace}
\noindent\(\fbox{$\text{Unresolved} \text{Dynamic} \text{Content}$}\)
\end{doublespace}

\begin{doublespace}
\noindent\(\pmb{\text{SeedRandom}[28];}\\
\pmb{\text{RandomSample}[\text{Range}[10,99],9]\text{//}\text{Print}}\)
\end{doublespace}

\noindent\(\{33,55,46,79,86,19,13,40,58\}\)

\end{document}
